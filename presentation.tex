\documentclass[xcolor=table]{beamer}
\usepackage{beamerthemecern}
\usepackage{booktabs}
\usepackage{pgffor}
\usepackage{lstlinebgrd}
\usepackage{enumitem}
\usepackage{caption}
\usepackage{listings}

\definecolor{lightblue}{gray}{0.95}

% Customize caption for listing removing "Listing X"
\DeclareCaptionFormat{listing}{\colorbox{lightblue}{\parbox{\textwidth - 2mm}{#3}}}
\captionsetup[lstlisting]{format=listing,font={scriptsize,sf},justification=centering}

% Zebra colored listings
\newcommand\realnumberstyle[1]{}

\makeatletter
\newcommand{\zebra}[3]{%
    {\realnumberstyle{#3}}%
    \begingroup
    \lst@basicstyle
    \ifodd\value{lstnumber}%
        \color{#1}%
    \else
        \color{#2}%
    \fi
        \rlap{\hspace*{\lst@numbersep}%
        \color@block{\linewidth}{\ht\strutbox}{\dp\strutbox}%
        }%
    \endgroup
}
\makeatother

% Listing customization
\lstset{
    language=Python,                % choose the language of the code
    basicstyle=\fontfamily{pcr}\tiny,       % the size of the fonts that are used for the code
    tabsize=2,
    captionpos=b,
    %numbers=left,
    stepnumber=1,
    numberstyle=\tiny,
    numbersep=5pt,
    frame=lines,
    escapeinside={£}{£},
    breaklines=true,
    keywordstyle=\color[rgb]{0,0,1},
    commentstyle=\color[rgb]{0.133,0.545,0.133},
    stringstyle=\color[rgb]{1,0,0}
}

\makeatletter
%%%%%%%%%%%%%%%%%%%%%%%%%%%%%%%%%%%%%%%%%%%%%%%%%%%%%%%%%%%%%%%%%%%%%%%%%%%%%%
%
% \btIfInRange{number}{range list}{TRUE}{FALSE}
%
% Test in int number <number> is element of a (comma separated) list of ranges
% (such as: {1,3-5,7,10-12,14}) and processes <TRUE> or <FALSE> respectively

\newcount\bt@rangea
\newcount\bt@rangeb

\newcommand\btIfInRange[2]{%
    \global\let\bt@inrange\@secondoftwo%
    \edef\bt@rangelist{#2}%
    \foreach \range in \bt@rangelist {%
        \afterassignment\bt@getrangeb%
        \bt@rangea=0\range\relax%
        \pgfmathtruncatemacro\result{ ( #1 >= \bt@rangea) && (#1 <= \bt@rangeb) }%
        \ifnum\result=1\relax%
            \breakforeach%
            \global\let\bt@inrange\@firstoftwo%
        \fi%
    }%
    \bt@inrange%
}
\newcommand\bt@getrangeb{%
    \@ifnextchar\relax%
        {\bt@rangeb=\bt@rangea}%
        {\@getrangeb}%
}
\def\@getrangeb-#1\relax{%
    \ifx\relax#1\relax%
        \bt@rangeb=100000%   \maxdimen is too large for pgfmath
    \else%
        \bt@rangeb=#1\relax%
    \fi%
}

\newcommand<>{\btLstHL}[1]{%
  \only#2{\btIfInRange{\value{lstnumber}}{#1}
        {\color{lightgray}\def\lst@linebackgroundcolor{\color@block}}
        {\def\lst@linebgrdcmd####1####2####3{}}}%
}%
\makeatother

\begin{document}
\title{Understanding applications performance}
\author{Marco Guerri}
\date{\today}

\frontcover
\frame{\titlepage}
\frame{\tableofcontents}
\section{Dirac Benchmark 2012}

\begin{frame}[fragile]{Dirac Benchmark 12}
    \begin{columns}[t]
        \begin{column}{0.5\textwidth}
            \setlist{leftmargin=1mm,label=$\diamond$}
            \begin{itemize}[noitemsep,topsep=0pt]
            \small
                \item Python benchmark, basically \textit{random.normalvariate()}
                 \item Considered an excellent predictor of performance of
                          LHCb MC jobs
            \end{itemize}
            \hypersetup{colorlinks=true, linkbordercolor=red, pdfborderstyle={/S/U/W 0.2}}
            \begin{lstlisting}[language=Python, caption=Source code avilable on
                                \href{https://github.com/DIRACGrid/DB12}{github},
                                linebackgroundcolor={\btLstHL<2>{6}}]
for i in range( iterations + 1):
    if i == 1:
      start = os.times()

    for _j in xrange( n ):
      t = random.normalvariate( 10, 1 )
      m += t
      m2 += t * t
      p += t
      p2 += t * t
            \end{lstlisting}
            \hypersetup{colorlinks=true, linkbordercolor=lightblue, pdfborderstyle={}}
        \end{column}
    \begin{column}{0.5\textwidth}
\def\arraystretch{1.2}
\arrayrulewidth=0.4pt

\begin{table}
\vspace{-7mm}
\scriptsize
\begin{tabular}{ l |  c |  c  }
   \multicolumn{3}{ c }{Hardware configuration} \\
   \hline
   & \textbf{Ivy Bridge} & \textbf{Haswell}  \\
  \hline
  CPU & E5-2650v2  & E5-2640v3 \\
%   & 8c/16t & 8c/16t  \\
%   & 2.6/3.4  & 2.6/3.4 \\
  RAM  & 64 GiB DDR3 & 128 GiB DDR4 \\
  OS & CentOS 7.3 & CentOS 7.3 \\
  \hline
\end{tabular}
\end{table}

\vspace{-5mm}
\begin{table}

\def\fourthrowcolor{}
\only<3>{\def\fourthrowcolor{\rowcolor{lightgray}}}

\scriptsize
\begin{tabular}{ l | c |  c |  c }
   \multicolumn{4}{ c }{Benchmark Results} \\
   \hline
   & \textbf{IB} & \textbf{HSW} & Speedup \\
  \hline
    $HS06_{32 t}$ & 350.19 & 360.87 & +3\%\\
    $HS06_{1 t}$ & 26.35 & 28.54 & +8\% \\
    $DB12_{32 t}$ & 10.98 & 12.27 & +12\% \\
\fourthrowcolor $DB12_{1 t}$ & 20.99 & 30.17 & +44\% \\
\hline
\end{tabular}
\end{table}

    \end{column}
\end{columns}
\end{frame}

\begin{frame}[fragile]{DB 12 - Functions profile}
\begin{lstlisting}[language=Python, caption=Functions profile on Ivy Bridge,
                   linebackgroundcolor={%
                        \btLstHL<2>{2}%
                        \btLstHL<3>{7}}]
#rank % cumulative%  function-name             image-name
0:  40.065  40.065  PyEval_EvalFrameEx        /lib64/libpython2.7.so.1.0
1:  10.345  50.410  _Py_add_one_to_index_C    /lib64/libpython2.7.so.1.0
2:   6.474  56.884  PyFloat_GetInfo           /lib64/libpython2.7.so.1.0
3:   3.743  60.627  _PyLong_Init              /lib64/libpython2.7.so.1.0
4:   3.600  64.227  PyFloat_FromDouble        /lib64/libpython2.7.so.1.0
5:   3.530  67.756  .text                     /usr/lib64/.../_randommodule.so
6:   3.089  70.845  Py_UniversalNewlineFread  /lib64/libpython2.7.so.1.0
7:   2.665  73.510  __ieee754_log_avx         /lib64/libm.so.6
8:   2.554  76.065  PyDict_GetItem            /lib64/libpython2.7.so.1.0
9:   2.273  78.337  _PyFloat_Unpack8          /lib64/libpython2.7.so.1.0
\end{lstlisting}

\vspace{-1mm}
\begin{lstlisting}[language=Python, caption=Functions profile on Haswell,
                   linebackgroundcolor={%
                        \btLstHL<2>{2}%
                        \btLstHL<3>{7}}]
#rank % cumulative%  function-name             image-name
0:  40.061  40.061  PyEval_EvalFrameEx        /lib64/libpython2.7.so.1.0
1:  10.344  50.406  _Py_add_one_to_index_C    /lib64/libpython2.7.so.1.0
2:   6.474  56.879  PyFloat_GetInfo           /lib64/libpython2.7.so.1.0
3:   3.743  60.622  _PyLong_Init              /lib64/libpython2.7.so.1.0
4:   3.599  64.222  PyFloat_FromDouble        /lib64/libpython2.7.so.1.0
5:   3.529  67.751  .text                     /usr/lib64/.../_randommodule.so
6:   3.089  70.840  Py_UniversalNewlineFread  /lib64/libpython2.7.so.1.0
7:   2.665  73.504  __ieee754_log_avx         /lib64/libm.so.6
8:   2.554  76.059  PyDict_GetItem            /lib64/libpython2.7.so.1.0
9:   2.273  78.331  _PyFloat_Unpack8          /lib64/libpython2.7.so.1.0
\end{lstlisting}

\end{frame}

\begin{frame}[fragile]{DB 12 - PyEval\_EvalFramEx}

\begin{columns}[t]
    
\begin{column}{0.5\textwidth}
\vspace{-5mm}
\setlist{leftmargin=1mm,label=$\diamond$}
\begin{itemize}
    \small
    \item Python source code is compiled into bytecode
    \item Bytecode is executed by interpreter (CPython)
\end{itemize}
\begin{lstlisting}[language=Python, caption=Function disassembly,
                   linebackgroundcolor={\btLstHL<2>{1-5}%
                                        \btLstHL<3>{6-17}%
                                        \btLstHL<3>{6-17}%
                                        \btLstHL<4>{12}}]
>>> def myfunc():
...     a = 120
...     b = a*10
...     return b
... 
>>> import dis
>>> dis.dis(myfunc)
  2           0 LOAD_CONST         1 (120)
              3 STORE_FAST         0 (a)

  3           6 LOAD_FAST          0 (a)
              9 LOAD_CONST         2 (10)
             12 BINARY_MULTIPLY     
             13 STORE_FAST         1 (b)

  4          16 LOAD_FAST          1 (b)
             19 RETURN_VALUE  
\end{lstlisting}
\end{column}

\begin{column}{0.5\textwidth}
\vspace{-5mm}
\begin{lstlisting}[language=c, caption=Excerpt from PyEval\_EvalFrameEx,
                   linebackgroundcolor={\btLstHL<4>{8}}]
switch (opcode) {
    case NOP:
        <NOP Instruction>
        break;
    case LOAD_FAST:
        <LOAD_FAST Instruction>
        break;
    case LOAD_CONST:
        <LOAD_CONST Instruction>
        break;
    case STORE_FAST:
        <STORE_FAST Instruction>
        break;
    case POP_TOP:
        <POP_TOP Instruction>
        break;
    case ROT_TWO:
        <ROT_TWO Instruction>
        break;
    [...]
    case UNARY_CONVERT:
        <UNARY_CONVERT Instruction>
        break;
    [..]
}
\end{lstlisting}
\end{column}
\end{columns}
\end{frame}

\begin{frame}[fragile]{DB 12 - Profiling}

\begin{table}
\scriptsize
\begin{tabular}{ l |  r |  c | c | c| c | }
   \multicolumn{6}{ c }{Performance statistics} \\
   \hline
   & \textbf{Ivy Bridge} & \% & \textbf{Haswell} & \%  & $\Delta$ \\
    \hline
  task-clock & 34725.557061  & & & \\
  context-switches & 1,757  & & & \\
  cpu-migrations & 0  & & & \\
  page-faults & 77,735  & & & \\
  cycles & 115,880,566,912  & & & \\
  stalled-cycles-frontend & 33,935,812,197  & & & \\
  instructions & 196,423,070,931  & & & \\
  branches & 39,007,221,715  & & & \\
  branch-misses & 771,858,119  & & & \\
  L1-dcache-loads & 65,738,912,645  & & & \\
  L1-dcache-load-misses & 223,445,275  & & & \\
  LLC-loads & 18,182,616  & & & \\
  \hline
\end{tabular}
\end{table}

% Ivy Bridge Profile
%      34725.557061      task-clock (msec)         #    1.000 CPUs utilized          
%             1,757      context-switches          #    0.051 K/sec                  
%                 0      cpu-migrations            #    0.000 K/sec                  
%            77,735      page-faults               #    0.002 M/sec                  
%   115,880,566,912      cycles                    #    3.337 GHz                      (44.45%)
%    33,935,812,197      stalled-cycles-frontend   #   29.29% frontend cycles idle     (44.45%)
%   196,423,070,931      instructions              #    1.70  insn per cycle         
%                                                  #    0.17  stalled cycles per insn  (55.56%)
%    39,007,221,715      branches                  # 1123.300 M/sec                    (55.56%)
%       771,858,119      branch-misses             #    1.98% of all branches          (55.56%)
%    65,738,912,645      L1-dcache-loads           # 1893.099 M/sec                    (44.44%)
%       223,445,275      L1-dcache-load-misses     #    0.34% of all L1-dcache hits    (22.22%)
%        18,182,616      LLC-loads                 #    0.524 M/sec                    (22.22%)
%        12,658,059      LLC-load-misses           #   69.62% of all LL-cache hits     (33.33%)
%      34.734003622 seconds time elapsed

\vspace{-5mm}
\begin{lstlisting}[language=c, caption=Excerpt from PyEval\_EvalFrameEx]
HoHoHo Merry christmas
\end{lstlisting}

\end{frame}
\backcover

\end{document}

